  %%%%%%%%%%%%%%%%%%%%%%%%%%%%%%%%%%%%% -*- coding: utf-8; mode: latex -*- %%
  %
%%%%%                  TÍTULO E DATA OFICIAL DA TESE
 %%%
  %

\def\date{September 2021}
\def\title{Explaining Parkinson’s Disease Computational Diagnostic based on Speech Analysis}

% hypernavigation in PDF docs
\hypersetup{
   debug=false,
   linkcolor=blue,  %%% cor do tableofcontents, \ref, \footnote, etc
   citecolor=blue,  %%% cor do \cite
   urlcolor=blue,   %%% cor do \url e \href
   bookmarksopen=true,
   pdftitle={\title},
   pdfauthor={Artur Oliveira Fortunato},
   pdfsubject={Master's Thesis},
   pdfkeywords={Machine Learning,Speech,Explainability,Interpretability}
}

  %%%%%%%%%%%%%%%%%%%%%%%%%%%%%%%%%%%%%%%%%%%%%%%%%%%%%%%%%%%%%%%%%%%%%%%%%%%%%
  %
%%%%%                          CAPA DA TESE
 %%%
  %

\thispagestyle{empty}

\begin{singlespace}
\vbox to\textheight{%
%--------------------------------------------------
\vskip-1.3in%---------- LOGO E NOME IST/UTL -------
%--------------------------------------------------
\hskip-17mm\vbox to50mm{
\vfil%
\begin{tabular}{l}
\includegraphics[width=9cm]{figs/preliminar/IST_A_CMYK_POS.pdf}
\end{tabular}
\vfil
\vfil
}%
%--------------------------------------------------
\vskip18mm%---------- FIGURAS DA CAPA -------------
%--------------------------------------------------
\vbox to25mm{\LARGE\sl
\vfil
%\centerline{\psfig{file=figs/preliminar/tarantula.eps,height=25mm}}
\vfil
}%
%--------------------------------------------------
\vskip6mm%---------- TÍTULO -----------------------
%--------------------------------------------------
\vbox to25mm{\LARGE\sf
\vfil
\begin{center}
\textbf\title
\end{center}
\vfil
}%
%--------------------------------------------------
\vskip10mm%---------- NOME E GRAU ACTUAL -----------
%--------------------------------------------------
\vbox to25mm{\large
\vfil
\begin{center}
{\Large\sf\textbf {Artur Oliveira Fortunato}}\\   % author's name
\end{center}
\vfil
}%
%--------------------------------------------------
\vskip8mm%---------- GRAU A OBTER -----------------
%--------------------------------------------------
\vbox to8mm{\large\sf
\vfil
\centerline{Thesis to obtain the Master of Science Degree in}
\vskip3mm
\centerline{\LARGE\textbf{Computer Science and Engineering}}
\vfil
}%
%--------------------------------------------------
\vskip15mm%---------- ORIENTADOR -------------------
%--------------------------------------------------
\vbox to8mm{\large\sf
\vfil
\begin{center}
\begin{tabular}{c}
Supervisors: Doctor David Manuel Martins de Matos\\
            Doctor Alberto Abad Gareta\\
\end{tabular}
\end{center}
\vfil
}%
%%--------------------------------------------------
%\vfil
% %--------------------------------------------------
\vskip15mm%---------- J�RI -------------------------
% %--------------------------------------------------
%\vbox{\Large%
%\vfil%
%\begin{center}
%{\Large\sf\textbf{Examination Committee}}\\
%\end{center}
%\vfil%
%}%

\vbox to30mm{\large\sf
\vfil
\begin{center}
{\Large\sf\textbf{Examination Committee}}\\
\quad\\
\begin{tabular}{c}
Chairperson: \\
Supervisor: Doctor David Manuel Martins de Matos\\
Member of the Committee: \\
\end{tabular}
\end{center}
\vfil
}%
%--------------------------------------------------
\vskip12mm%---------- DATA -------------------------
%--------------------------------------------------
\begin{center}
{\Large\sf\textbf\date}
\end{center}

%--------------------------------------------------
}%vbox
\end{singlespace}
\newpage

  %%%%%%%%%%%%%%%%%%%%%%%%%%%%%%%%%%%%%%%%%%%%%%%%%%%%%%%%%%%%%%%%%%%%%%%%%%%%%
  %
%%%%%                             AGRADECIMENTOS
 %%%
  %

\chapter*{Agradecimentos}
%\chapter*{Acknowledgements}
\thispagestyle{empty}

Queria começar por agradecer aos meus orientadores, professores David Matos e Alberto Abad, pelo apoio durante estes meses de trabalho. Todos as sugestões e revisões foram fundamentais para a ideação e desenvolvimento deste trabalho. Agradeço também à engenheira Catarina Botelho, pelas ideias e discussões que me ajudaram a definir a minha dissertação. Quero agradecer também ao engenheiro André Gonçalves pelas suas revisões sobre esta dissertação.

Gostaria de agradecer à minha namorada, Luísa Pereira, por toda a motivação que me deu durante esta etapa fundamental da minha educação. Por fim, quero agradecer aos meus pais, André Fortunato e Anabela Oliveira, pelo incansável apoio e dedicação que dispuseram, além das múltiplas revisões deste trabalho, que em muito contribuiram para o seu resultado final.

\vfill
\begin{flushright}
  \begin{minipage}{8cm}
    \begin{center}
      Lisboa, \today

      Artur Oliveira Fortunato
    \end{center}
  \end{minipage}
\end{flushright}

\cleardoublepage

  %%%%%%%%%%%%%%%%%%%%%%%%%%%%%%%%%%%%%%%%%%%%%%%%%%%%%%%%%%%%%%%%%%%%%%%%%%%%%
  %
%%%%%                            DEDICATÓRIAS
 %%%
  %

\chapter*{}
\thispagestyle{empty}

% DEDICAR!
\vfill
\mbox{}
\vfill\Large
\begin{flushright}
  \begin{minipage}{8cm}
    \begin{center}
	Para os meus pais, 
	
	André e Anabela
    \end{center}
  \end{minipage}
\end{flushright}
\normalsize\vfill

\cleardoublepage

  %%%%%%%%%%%%%%%%%%%%%%%%%%%%%%%%%%%%%%%%%%%%%%%%%%%%%%%%%%%%%%%%%%%%%%%%%%%%%
  %
%%%%%                                RESUMO
 %%%
  %

\chapter*{Resumo}
\thispagestyle{empty}

A Doença de Parkinson (PD) é uma doença neurodegenerativa que afecta o sistema nervoso central. A doença manifesta-se na fala do paciente, que normalmente se torna desarticulada, monotónica, e ofegante. Estes sintomas fornecem um poderoso biomarcador para a detecção de PD.

O presente estudo incluiu dois objectivos. Primeiro, analisar o desempenho de um modelo independente de língua para o diagnóstico de PD. Para este trabalho, foram utilizados três \textit{datasets} de línguas diferentes. Uma abordagem de base (com um modelo treinado e testado com discurso na mesma língua) alcançou uma precisão máxima de 90\%. Foi também executado um passo intermédio, onde um modelo foi treinado com discurso numa língua e parte de um \textit{dataset} numa língua diferente e testado com a restante parte do segundo \textit{dataset}. O desempenho deste modelo semi independente da língua foi semelhante ao desempenho da abordagem de base. Estes resultados demonstraram a capacidade do nosso modelo de ser re-treinado com novos dados de uma nova língua e de ser estendido a pacientes que falam a nova língua. Em seguida, o modelo independente da língua foi treinado, atingindo uma \textit{accuracy} máxima de 67\% e um valor de \textit{recall} de 76\%. Enquanto a \textit{accuracy} do nosso modelo é menor do que o estado da arte (77\%), a \textit{recall}, que representa a capacidade de detectar pacientes com PD, é bastante superior ao melhor trabalho anterior (53\%). Em segundo lugar, o modelo de explicabilidade LIME foi utilizado para explicar cada diagnóstico produzido pelo modelo de classificação. O relatório inclui a probabilidade do sujeito pertencer a cada classe (PD ou grupo de controlo) e as cinco principais características com a maior contribuição para a classificação do modelo. Para cada característica, são fornecidos o valor médio, a gama de valores de um indivíduo saudável, a sua contribuição para a classificação, e uma pequena descrição. Estas informações permitem ao médico ter uma melhor compreensão da classificação do modelo, proporcionando assim uma maior confiança no mesmo. Uma avaliação da contribuição global de cada característica concluiu que tanto os MFCC como PLP fornecem aos modelos informações mais relevantes quando comparados com F0, HNR, jitter e shimmer.

Este trabalho contribuiu para aumentar a utilidade dos modelos de aprendizagem de máquinas para a detecção automática de PD. A sua contribuição incluiu uma nova abordagem para aplicação universal a qualquer língua e modelação independente da mesma. Além disso, o modelo de explicabilidade aplicado facilita a compreensão e promove a adopção de modelos computacionais de diagnóstico PD na prática médica real.
\newpage

  %%%%%%%%%%%%%%%%%%%%%%%%%%%%%%%%%%%%%%%%%%%%%%%%%%%%%%%%%%%%%%%%%%%%%%%%%%%%%
  %
%%%%%                            ABSTRACT
 %%%
  %

\chapter*{Abstract}
\thispagestyle{empty}

%Parkinson’s Disease (PD) is a neurodegenerative disorder that affects the central nervous system. One of the disease's manifestation is in the patient’s speech, which usually becomes slurred, monotonic, and breathy. These symptoms provide a powerful bio-marker for the detection of PD.
%The present work analyzes the subject’s speech, representing it through a set of acoustic features. With this representation, a Machine Learning (ML) model will be trained to diagnose PD. By performing cross-language tests on this classification model, we will evaluate the hypothesis that a ML classifier can correctly diagnose PD. In addition, this diagnosis will be independent of the language spoken by the subject. Furthermore, we will explore the use of an explainability model, which will “translate” the classification model’s diagnosis to a medical professional. The model will provide essential information for the clinician to trust and use this tool. Despite the good results achieved by many ML classification models, their acceptance for the diagnosis of this disease has not yet been achieved. Clinicians are unable to use the model’s diagnosis, as it lacks a medical-oriented interpretation. Therefore, this work is motivated by the practical need of a universal, language-independent model that can provide enough human-understandable information to support clinical usage of ML models on PD diagnosis. This project will provide a tool that can boost the transfer of such models from test environments to real-life usage.


Parkinson’s Disease (PD) is a neurodegenerative disorder that affects the central nervous system. One of the disease's manifestation is in the patient’s speech, which usually becomes slurred, monotonic, and breathy. These symptoms provide a powerful biomarker for the detection of PD.

The present work had two objectives. First, we aimed at assessing the performance of a language-independent model for the \gls{pd} diagnostic task. For this work, three datasets from different languages were used. A baseline approach (a model trained and tested with speech from the same language) achieved a maximum accuracy of 90\%. An intermediate step was also taken, where a model was trained with speech from one language and part of the speech contained in a different dataset (in a different language) and tested with the remaining part of the speech from the second dataset. This semi language-independent model's performance was similar to the baseline's performance. These results demonstrated the ability of our model to be re-trained with new data from a new language and be extended to patients speaking the new language. Next, the language-independent model was trained, reaching a maximum accuracy of 67\% and a recall value of 76\%. Although the accuracy of our model is lower than the state-of-the-art (77\%), the recall, which represents the capacity to detect \gls{pd} patients, is far better than the best previous work (53\%). Second, the LIME explainability model was used to generate an explanation report for each diagnostic produced by the classification model. The report includes the probability of a subject belonging to each class (\gls{pd} or \gls{hc}) and the top five features with the highest contribution to the model's classification. Each feature includes the average value, the range of values of a healthy individual, its contribution weight to the classification, and a small description. This information helps the clinician to understand the computational diagnostic, thus providing enhanced trust in the model. An evaluation of the global contribution of each feature concluded that both MFCC and PLP features provide the models with more relevant information than \gls{f0}, \gls{hnr}, jitter, and shimmer.

This work contributed to increase the usefulness of machine learning models for the automatic \gls{pd} detection. Its contribution was a \gls{pd} detection approach that can be extended to any language. Furthermore, the explainability model applied herein facilitates the understanding and fosters the adoption of \gls{pd} diagnostic computational models in real medical practice.

\newpage

  %%%%%%%%%%%%%%%%%%%%%%%%%%%%%%%%%%%%%%%%%%%%%%%%%%%%%%%%%%%%%%%%%%%%%%%%%%%%%
  %
%%%%%                 FICHA BIBLIOGRAFICA -- PALAVRAS CHAVE
 %%%
  %

\chapter*{Palavras Chave \\ Keywords}
\thispagestyle{empty}

\section*{Palavras Chave}
{\large % EM PORTUGUÊS

\noindent Aprendizagem de máquina

\noindent Fala

\noindent Explicabilidade

\noindent Interpretabilidade

}

\section*{Keywords}

{\large % EM INGLÊS

\noindent Machine Learning

\noindent Speech

\noindent Explainability

\noindent Interpretability

}

\vfill
%LATEX2HTML}

\cleardoublepage


  %%%%%%%%%%%%%%%%%%%%%%%%%%%%%%%%%%%%%%%%%%%%%%%%%%%%%%%%%%%%%%%%%%%%%%%%%%%%%
  %
%%%%%                           CHANGE OF NUMBERING
 %%%
  %

\pagestyle{plain}
\pagenumbering{roman}

  %%%%%%%%%%%%%%%%%%%%%%%%%%%%%%%%%%%%%%%%%%%%%%%%%%%%%%%%%%%%%%%%%%%%%%%%%%%%%
  %
%%%%%                               INDICES
 %%%
  %

% ``Table of contents'' (Apêndice).

\def\contentsname{Table of Contents}
\tableofcontents
\newpage

% Lista de figuras.
\listoffigures
\newpage

% Lista de tabelas.
\listoftables

% Does it always work? I expect so...
\cleardoublepage

  %
 %%%
%%%%%                            T H E    E N D
  %
  %%%%%%%%%%%%%%%%%%%%%%%%%%%%%%%%%%%%%%%%%%%%%%%%%%%%%%%%%%%%%%%%%%%%%%%%%%%%%

% Local Variables: 
% mode: latex
% TeX-master: "tese"
% End: 
