  %%%%%%%%%%%%%%%%%%%%%%%%%%%%%%%%%%%%% -*- coding: utf-8; mode: latex -*- %%
  %
%%%%%                  TÍTULO E DATA OFICIAL DA TESE
 %%%
  %

\def\date{September 2021}
\def\title{Explaining Parkinson’s Disease Computational Diagnostic based on Speech Analysis}

% hypernavigation in PDF docs
\hypersetup{
   debug=false,
   linkcolor=blue,  %%% cor do tableofcontents, \ref, \footnote, etc
   citecolor=red,  %%% cor do \cite
   urlcolor=blue,   %%% cor do \url e \href
   bookmarksopen=true,
   pdftitle={\title},
   pdfauthor={Author's Name},
   pdfsubject={Labore et Dolore},
   pdfkeywords={Labore, Dolore}
}

  %%%%%%%%%%%%%%%%%%%%%%%%%%%%%%%%%%%%%%%%%%%%%%%%%%%%%%%%%%%%%%%%%%%%%%%%%%%%%
  %
%%%%%                          CAPA DA TESE
 %%%
  %

\thispagestyle{empty}

\begin{singlespace}
\vbox to\textheight{%
%--------------------------------------------------
\vskip-1.3in%---------- LOGO E NOME IST/UTL -------
%--------------------------------------------------
\hskip-17mm\vbox to50mm{
\vfil%
\begin{tabular}{l}
\includegraphics[width=9cm]{figs/preliminar/IST_A_CMYK_POS.pdf}
\end{tabular}
\vfil
\vfil
}%
%--------------------------------------------------
\vskip18mm%---------- FIGURAS DA CAPA -------------
%--------------------------------------------------
\vbox to25mm{\LARGE\sl
\vfil
%\centerline{\psfig{file=figs/preliminar/tarantula.eps,height=25mm}}
\vfil
}%
%--------------------------------------------------
\vskip6mm%---------- TÍTULO -----------------------
%--------------------------------------------------
\vbox to25mm{\LARGE\sf
\vfil
\begin{center}
\textbf\title
\end{center}
\vfil
}%
%--------------------------------------------------
\vskip10mm%---------- NOME E GRAU ACTUAL -----------
%--------------------------------------------------
\vbox to25mm{\large
\vfil
\begin{center}
{\Large\sf\textbf {Artur Oliveira Fortunato}}\\   % author's name
\end{center}
\vfil
}%
%--------------------------------------------------
\vskip8mm%---------- GRAU A OBTER -----------------
%--------------------------------------------------
\vbox to8mm{\large\sf
\vfil
\centerline{Thesis to obtain the Master of Science Degree in}
\vskip3mm
\centerline{\LARGE\textbf{Computer Science and Engineering}}
\vfil
}%
%--------------------------------------------------
\vskip15mm%---------- ORIENTADOR -------------------
%--------------------------------------------------
\vbox to8mm{\large\sf
\vfil
\begin{center}
\begin{tabular}{c}
Supervisors: Doctor David Manuel Martins de Matos\\
            Doctor Alberto Abad Gareta\\
\end{tabular}
\end{center}
\vfil
}%
%%--------------------------------------------------
%\vfil
% %--------------------------------------------------
\vskip15mm%---------- J�RI -------------------------
% %--------------------------------------------------
%\vbox{\Large%
%\vfil%
%\begin{center}
%{\Large\sf\textbf{Examination Committee}}\\
%\end{center}
%\vfil%
%}%

\vbox to30mm{\large\sf
\vfil
\begin{center}
{\Large\sf\textbf{Examination Committee}}\\
\quad\\
\begin{tabular}{c}
Chairperson: Doctor name-of-president\\
Supervisor: Doctor David Manuel Martins de Matos\\
Member of the Committee: Doctor name-of-member-of-committee\\
\end{tabular}
\end{center}
\vfil
}%
%--------------------------------------------------
\vskip12mm%---------- DATA -------------------------
%--------------------------------------------------
\begin{center}
{\Large\sf\textbf\date}
\end{center}

%--------------------------------------------------
}%vbox
\end{singlespace}
\newpage

  %%%%%%%%%%%%%%%%%%%%%%%%%%%%%%%%%%%%%%%%%%%%%%%%%%%%%%%%%%%%%%%%%%%%%%%%%%%%%
  %
%%%%%                             AGRADECIMENTOS
 %%%
  %

\chapter*{Agradecimentos}
%\chapter*{Acknowledgements}
\thispagestyle{empty}

% AGRADECER!

Magníficos Agradecimentos

\vfill
\begin{flushright}
  \begin{minipage}{8cm}
    \begin{center}
      Lisboa, \today

      Artur Oliveira Fortunato
    \end{center}
  \end{minipage}
\end{flushright}

\cleardoublepage

  %%%%%%%%%%%%%%%%%%%%%%%%%%%%%%%%%%%%%%%%%%%%%%%%%%%%%%%%%%%%%%%%%%%%%%%%%%%%%
  %
%%%%%                            DEDICATÓRIAS
 %%%
  %

\chapter*{}
\thispagestyle{empty}

% DEDICAR!
\vfill
\mbox{}
\vfill\Large
\begin{flushright}
  \begin{minipage}{8cm}
    \begin{center}

Dedicatória interessante

    \end{center}
  \end{minipage}
\end{flushright}
\normalsize\vfill

\cleardoublepage

  %%%%%%%%%%%%%%%%%%%%%%%%%%%%%%%%%%%%%%%%%%%%%%%%%%%%%%%%%%%%%%%%%%%%%%%%%%%%%
  %
%%%%%                                RESUMO
 %%%
  %

\chapter*{Resumo}
\thispagestyle{empty}

[TRADUZIR PARA PORTUGUÊS] Parkinson’s Disease (PD) is a neurodegenerative disorder that affects the central nervous system. The disease manifests
itself in the patient’s speech, which usually becomes slurred, monotonic, and breathy. This symptoms provide a powerful
biomarker for the detection of PD.
The present work will analyse the subject’s speech, representing it with a set of acoustic features. With this repre-
sentation, a Machine Learning (ML) model will be trained to diagnose PD. By performing cross-language tests on this
classification model, we will evaluate the hypothesis that a ML classifier can correctly diagnose PD. In addition, this
diagnosis will be independent of the language spoken by the subject. Furthermore, we will explore the use of an explain-
ability model, which will “translate” the classification model’s diagnosis to a medical professional. The model will provide
essential information for the clinician to trust and use this tool.
Despite the good results achieved by many ML classification models, their acceptance for the diagnosis of this disease
has not yet been achieved. Clinicians are unable to use the model’s diagnosis, as it lacks a medical-oriented interpretation.
Therefore, this work is motivated by the practical need of a universal, language-independent model that can provide
enough human-understandable information to support clinical usage of ML models on PD diagnosis. This project will
provide a tool that can boost the transfer of such models from test environments to real-life usage.
\newpage

  %%%%%%%%%%%%%%%%%%%%%%%%%%%%%%%%%%%%%%%%%%%%%%%%%%%%%%%%%%%%%%%%%%%%%%%%%%%%%
  %
%%%%%                            ABSTRACT
 %%%
  %

\chapter*{Abstract}
\thispagestyle{empty}

%Parkinson’s Disease (PD) is a neurodegenerative disorder that affects the central nervous system. One of the disease's manifestation is in the patient’s speech, which usually becomes slurred, monotonic, and breathy. These symptoms provide a powerful bio-marker for the detection of PD.
%The present work analyzes the subject’s speech, representing it through a set of acoustic features. With this representation, a Machine Learning (ML) model will be trained to diagnose PD. By performing cross-language tests on this classification model, we will evaluate the hypothesis that a ML classifier can correctly diagnose PD. In addition, this diagnosis will be independent of the language spoken by the subject. Furthermore, we will explore the use of an explainability model, which will “translate” the classification model’s diagnosis to a medical professional. The model will provide essential information for the clinician to trust and use this tool. Despite the good results achieved by many ML classification models, their acceptance for the diagnosis of this disease has not yet been achieved. Clinicians are unable to use the model’s diagnosis, as it lacks a medical-oriented interpretation. Therefore, this work is motivated by the practical need of a universal, language-independent model that can provide enough human-understandable information to support clinical usage of ML models on PD diagnosis. This project will provide a tool that can boost the transfer of such models from test environments to real-life usage.


Parkinson’s Disease (PD) is a neurodegenerative disorder that affects the central nervous system. One of the disease's manifestation is in the patient’s speech, which usually becomes slurred, monotonic, and breathy. These symptoms provide a powerful biomarker for the detection of PD. \\
The present work comprised two objectives. First, to analyze the performance of a language-independent model for the \gls{pd} diagnostic task. For this work, three datasets were used (one in European Portuguese, one in European Spanish, and one in European English). A baseline approach (a model trained and tested with speech from the same language) achieved a maximum accuracy of 90\%. An intermediate step was also taken, where a model was trained with speech from one language and part of the speech contained in a different dataset (in a different language) and tested with the remaining part of the speed from the second dataset. This semi language-independent model's performance was similar to the baseline's performance. These results demonstrated the ability of our model to be re-trained with new data from a new language and be extended to patients speaking the new language. Next, the language-independent model was trained, reaching a maximum accuracy of 67\% and a recall value of 76\%. While the accuracy of our model is smaller than the state-of-the-art (77\%), the recall, which represents the capacity to detect \gls{pd} patients, is far better than the best previous work (53\%). Second, the LIME explainability model was used to generate an explanation report for each diagnostic produced by the classification model. The report includes the probability of each class (\gls{pd} or HC) and the top five features with the highest contribution to the model's classification. For each feature, the average value, the range of values of a healthy individual, its contribution to the classification, and a small description are provided. Such report allows for the clinician to have a more complete understanding of the computational classification, thus providing enhanced trust in the model. An evaluation of the global contribution of each feature concluded that both MFCC and PLP parameters provide the models with more relevant information when compared to \gls{f0}, HNR, Jitter, and Shimmer.\\
This work contributed towards increasing the usefulness of machine learning models towards automatic \gls{pd} detection. Its contribution included a new approach for universal application to any language and language-independent modeling. Furthermore, the applied explainability model facilitates the adoption of \gls{pd} diagnostic computational models in real medical practice.

\newpage

  %%%%%%%%%%%%%%%%%%%%%%%%%%%%%%%%%%%%%%%%%%%%%%%%%%%%%%%%%%%%%%%%%%%%%%%%%%%%%
  %
%%%%%                 FICHA BIBLIOGRAFICA -- PALAVRAS CHAVE
 %%%
  %

\chapter*{Palavras Chave \\ Keywords}
\thispagestyle{empty}

\section*{Palavras Chave}
{\large % EM PORTUGUÊS

\noindent Aprendizagem de máquina

\noindent Discurso

\noindent Explicabilidade

\noindent Interpretabilidade

}

\section*{Keywords}

{\large % EM INGLÊS

\noindent Machine Learning

\noindent Speech

\noindent Explainability

\noindent Interpretability

}

\vfill
%LATEX2HTML}

\cleardoublepage


  %%%%%%%%%%%%%%%%%%%%%%%%%%%%%%%%%%%%%%%%%%%%%%%%%%%%%%%%%%%%%%%%%%%%%%%%%%%%%
  %
%%%%%                           CHANGE OF NUMBERING
 %%%
  %

\pagestyle{plain}
\pagenumbering{roman}

  %%%%%%%%%%%%%%%%%%%%%%%%%%%%%%%%%%%%%%%%%%%%%%%%%%%%%%%%%%%%%%%%%%%%%%%%%%%%%
  %
%%%%%                               INDICES
 %%%
  %

% ``Table of contents'' (Apêndice).

\def\contentsname{Table of Contents}
\tableofcontents
\newpage

% Lista de figuras.
\listoffigures
\newpage

% Lista de tabelas.
\listoftables

% Does it always work? I expect so...
\cleardoublepage

  %
 %%%
%%%%%                            T H E    E N D
  %
  %%%%%%%%%%%%%%%%%%%%%%%%%%%%%%%%%%%%%%%%%%%%%%%%%%%%%%%%%%%%%%%%%%%%%%%%%%%%%

% Local Variables: 
% mode: latex
% TeX-master: "tese"
% End: 
