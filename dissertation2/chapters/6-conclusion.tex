  %%%%%%%%%%%%%%%%%%%%%%%%%%%%%%%%%%%%%%% -*- coding: utf-8; mode: latex -*- %%
  %
%%%%%                       CHAPTER
 %%%
  %

\chapter{Conclusions}
%\addcontentsline{lof}{chapter}{\thechapter\quad Nihil Molestiae}
%\addcontentsline{lot}{chapter}{\thechapter\quad Nihil Molestiae}
\label{ch:magna}

%\begin{quotation}
%  {\small\it Neque porro quisquam est qui dolorem ipsum quia dolor sit amet, consectetur, adipisci velit...}

%{\small\it -- Cerico}
%\end{quotation}
This work focused on two problems on the \gls{pd} diagnosis task: universality and explainability. \\
First, we evaluated the performance of a language-independent model for the \gls{pd} diagnosis task. Baseline results (training and testing the model with subjects speaking the same language) achieved a maximum accuracy of 90\% with both neural network architectures tested. An intermediate step was taken between the baseline and a language-independent model, where a model was trained with one dataset and with 90\% of the speakers of another dataset (with different language speakers) and tested with the remaining 10\% of the second dataset. Both architectures yielded a maximum accuracy of 90\%, without losing performance compared to the baseline. This demonstrated the capacity of these models to adapt to a new language with a reduced amount of training data when pre-trained with a different language. This can be useful as lack of training data is usually a problem for these experiments. Although the results were very positive, it is worth noticing that the percentage of the new dataset used for training (90\%) is still high. Reducing the amount of data to re-train a model can be worth investigating. When training a language-independent model (tested with two datasets and tested with the third), accuracy dropped to 67\% (architecture 1) and 66\% (architecture 2). Although results were inferior to the state-of-the-art regarding accuracy - similar work achieved an accuracy of 77\% with a language-independent model \cite{parkinson_three_languages} - our model with highest accuracy yielded a maximum value of 76\% on the recall metric, significantly higher than the 53\% achieved by the other work, thus producing less false negatives (\gls{pd} subjects classified as \gls{hc}). \\
Second, LIME model was used to build an explanation for each diagnostic. This step allowed to explain the classification result in a human-understandable way to the medical professional, thus providing trust in the model. This allows computational diagnostic models to be used in clinical scenarios, as they often achieve an inferior error rate compared to human diagnostic. The report contained the probability of the subject belonging to each class (\gls{pd} and \gls{hc}), as well as the top five acoustic features with higher contribution to the model's classification. For each feature, the contribution weight and the subject's average value were also included, together with the range of values of a healthy patient and a small description of the feature. This report largely extends the information produced by the classification model, which only exposes the final diagnostic, thus providing the medical professional with information that allows it to make an informed diagnostic. Finally, a global analysis was conducted to evaluate the average contribution of each acoustic feature extracted and the percentage of test subjects for which each feature was in the top five with higher contribution. Combining both results, we concluded that MFCC and PLP features represent better information for this task than \gls{f0}, jitter, shimmer, and HNR.
It is important to note that both MFCC and PLP are abstract mathematical representations of sound, and are therefore difficult to explain to a medical professional. Additionally, to the best of our knowledge, there is no known range of values considered define a healthy patient, which prevents to generate a complete report on these features. It is important to extend this work with simpler, easy to understand features in order for the model to be used in a clinical scenario. Finally, although the report provides with multiple insights on the classification process, no clinical tests were conducted during this work. \\
There are a number of paths to continue this work. \\
First, the current pipeline presents some limitations that should be addressed. As previously described, there are complexity limitations associated with abstract features, such as \gls{plp}s and \gls{mfcc}s. By using simpler features, such as Logarithmic Filter Banks (instead of \gls{mfcc}) would increase the ability of the medical professional to understand, and therefore trust, the model's output. In addition, graphical representations of the physical manifestation of each feature can be added to the explanation. The normal values for some features, such as \gls{f0}, depend on meta features (the normal values for \gls{f0} for males is between 105 and 160 Hz, while for females it's 175 to 245 \textit{Hz}). Thus, adding the gender as a feature for the model would provide important information which could help improve the model's performance. \\
Both the classification and explanation pipeline's steps have further aspects to be explored. Namely, the similarity between the average contribution (weight) value of each feature on the explanation model suggested some correlation between features. This hypothesis can be further studied, using a model to evaluate the interactions between features, such as factorization machines. Detecting redundant features could help reduce the model's complexity, thus reducing resource necessity. Also, the results achieved on the semi language-independent experiments showed that there was no performance loss when training a model with two languages. Further analysis on the impact of varying the training percentage of the test language would shed light into the relation between data quantity used to re-train a model and the eventual performance loss. Moreover, this work focuses on explaining the diagnosis of each patient. A study on the global contributions of each feature could clarify the their individual importance to the \gls{pd} classification task. This could be done by using LIME to generate global explanations', or by using models such as \gls{cav}s \cite{TCAV} or \gls{nam}s \cite{NAM}. Finally, both for the classification and explanations' steps, different models can be used to make a comparative analysis. This would allow to both assess the classification ability of multiple models, but also to compare the explanations generated by various models and the trust provided to the medical professionals. \\
The goal of generating explanations is to provide the medical professionals with a tool that can shed light into the \textit{black-box} classification models. Thus, these models should be tested in real-world scenarios, to rate their adequacy to perform this task. During the real-world evaluation, a comparative analysis could be conducted between explainability models, in order to assess which ones provide more trust to the end users for the product (the medical professionals). This can be done by generating explanations for the same user using different explainability models and assessing the degree of confidence of the medical professional in each one of them. This evaluation could also lead to the conclusion that a combination of both methods provides more information, which would provide a higher level of trust by the medical professional on the classification models. Feature types (such as audio ou images) should also be compared, as to understand which are more adequate to be used by medical professionals. For example, the explanations generated by the model developed during this work could be compared with ones produced by the work described on section 3.3, in which LIME was used to explain \gls{pd} diagnostic with SPECT DaTSCAN images of the brain. 
  %
 %%%
%%%%%                           THE END
  %
  %%%%%%%%%%%%%%%%%%%%%%%%%%%%%%%%%%%%%%%%%%%%%%%%%%%%%%%%%%%%%%%%%%%%%%%%%%%%%

%%% Local Variables: 
%%% mode: latex
%%% TeX-master: "tese"
%%% End: 
