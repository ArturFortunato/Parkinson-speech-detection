  %%%%%%%%%%%%%%%%%%%%%%%%%%%%%%%%%%%%%%% -*- coding: utf-8; mode: latex -*- %%
  %
%%%%%                       CHAPTER
 %%%
  %

\chapter{Conclusions}
%\addcontentsline{lof}{chapter}{\thechapter\quad Nihil Molestiae}
%\addcontentsline{lot}{chapter}{\thechapter\quad Nihil Molestiae}
\label{ch:magna}

%\begin{quotation}
%  {\small\it Neque porro quisquam est qui dolorem ipsum quia dolor sit amet, consectetur, adipisci velit...}

%{\small\it -- Cerico}
%\end{quotation}

\textbf{Future work: }\\
 - Add meta features, such as gender, that condition the normal value of a feature (such as F0) \\
 - Extend this work to generate global explanations at the same time \\
 - Test simpler features, such as log-filter bank (instead of MFCC) and compare accuracies. These would be easier to explain \\
 - Test the method in a real-world environment to assess the 
 - During the test, other explainability methods, such as the ones presented on papers X and Y (LIME on brain). This will shed light to the method preferred by medical professionals, and may lead to the conclusion that a combination of both methods provides more information, which will provide a higher level of trust by the medical professional on the models.\\
 - Nas experiências semi-independentes, reduzir a percentagem de treino do segundo dataset para encontrar o volume de dados necessário para re-adaptar um modelo numa língua para outra \\
 - generate global explanations
  %
 %%%
%%%%%                           THE END
  %
  %%%%%%%%%%%%%%%%%%%%%%%%%%%%%%%%%%%%%%%%%%%%%%%%%%%%%%%%%%%%%%%%%%%%%%%%%%%%%

%%% Local Variables: 
%%% mode: latex
%%% TeX-master: "tese"
%%% End: 
