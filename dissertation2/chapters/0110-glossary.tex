  %%%%%%%%%%%%%%%%%%%%%%%%%%%%%%%%%%%%%%% -*- coding: utf-8; mode: latex -*- %%
  %
%%%%%                     NOMENCLATURE / GLOSSARY
 %%%
  %

% a ordem não é relevante para o processamento, mas é-o para a gestão do
% conteúdo deste ficheiro.
% ATENÇÃO: maiúsculas e minúsculas são consideradas iguais.

  %%%%%%%%%%%%%%%%%%%%%%%%%%%%%%%%%%%%%%%%%%%%%%%%%%%%%%%%%%%%%%%%%%%%%%%%%%%%%
  %
%%%%%                     A  A  A  A  A  A  A  A  A
 %%%
  %

\nomenclature{AFS}{Andrew File System ou Advanced File
System. O Andrew File System é um sistema de
ficheiros distribuído. Foi inicialmente desenvolvido na Universidade de
Carnegie Mellon. O AFS apresenta várias vantagens sobre outros sistemas de
ficheiros, particularmente no que respeita às áreas de segurança e
escalabilidade.}

\nomenclature{AUTHOR}{\AUTHOR: arquitecura de
  geração de prosa narrativa. Ver a entrada
  correspondente a \StoryBook.}

  %%%%%%%%%%%%%%%%%%%%%%%%%%%%%%%%%%%%%%%%%%%%%%%%%%%%%%%%%%%%%%%%%%%%%%%%%%%%%
  %
%%%%%                     B  B  B  B  B  B  B  B  B
 %%%
  %

\nomenclature{BLARK}{Basic LAnguage Resource Kit.}

  %%%%%%%%%%%%%%%%%%%%%%%%%%%%%%%%%%%%%%%%%%%%%%%%%%%%%%%%%%%%%%%%%%%%%%%%%%%%%
  %
%%%%%                     C  C  C  C  C  C  C  C  C
 %%%
  %

  %%%%%%%%%%%%%%%%%%%%%%%%%%%%%%%%%%%%%%%%%%%%%%%%%%%%%%%%%%%%%%%%%%%%%%%%%%%%%
  %
%%%%%                     D  D  D  D  D  D  D  D  D
 %%%
  %

\nomenclature{DCR}{Data Category
Registry. A \DCR{} é
um componente do Linguistic Annotation
Framework que contém um
conjunto de categorias linguísticas definido
formalmente.}

\nomenclature{DCS}{Data Category Selection:
subconjuntos de uma \DCR{} que reflectem vários domínios temáticos e várias
classes e funções de categorias de dados.}

%--------------------------------------------------
% sec:content-determination
%
\nomenclature{Determinação de conteúdo}{Tarefa que decide que in\-for\-ma\-ção
  deve ser comunicada no documento de saída. Pode ser vista como o aspecto de
  conteúdo do planeador de documentos. No contexto do projecto \RAGS,
  esta designação corresponde a toda a fase de planeamento do
  documento.}

\nomenclature{DocuPlanner}{\docuplanner{} -- Um sistema de preparação de rascunhos de documentos.}

\nomenclature{DOM}{Document Object Model. O Document Object
Model é uma interface neutra relativamente a plataformas ou linguagens
particulares. Esta interface permite acesso dinâmico ao conteúdo, estrutura
e estilo de documentos que sigam este padrão.}

  %%%%%%%%%%%%%%%%%%%%%%%%%%%%%%%%%%%%%%%%%%%%%%%%%%%%%%%%%%%%%%%%%%%%%%%%%%%%%
  %
%%%%%                     E  E  E  E  E  E  E  E  E
 %%%
  %

\nomenclature{Edite}{Sistema desenvolvido para aceder em
  linguagem natural a uma base de dados de recursos turísticos dos quiosques
  multimédia do \INESC. O processo de acesso contempla três etapas: (a)
  análise morfológica responsável pela associação de informação morfo\pdash{}sintáctica às
  palavras da frase; (b) análise sintáctica (algoritmo de Earley), fase em que
  são geradas uma ou mais árvores sintácticas representantes da estrutura da
  frase; (c) análise semântica, onde é criada uma forma lógica que exprime o
  significado da frase em tratamento. O sistema é multilingue, suportando
  interacções em \langpt, \langes, \langen{} e \langfr.}

  %%%%%%%%%%%%%%%%%%%%%%%%%%%%%%%%%%%%%%%%%%%%%%%%%%%%%%%%%%%%%%%%%%%%%%%%%%%%%
  %
%%%%%                     F  F  F  F  F  F  F  F  F
 %%%
  %

\nomenclature{FUF}{Iniciais de \textsl{Funcional Unification
    For\-mal\-ism}.}

\nomenclature{Função de correspondência semântica}{No contexto da arquitectura
  dos dados,
  função que traduz a semântica dos dados que fluem através de uma ligação
  entre dois módulos em comunicação.}

  %%%%%%%%%%%%%%%%%%%%%%%%%%%%%%%%%%%%%%%%%%%%%%%%%%%%%%%%%%%%%%%%%%%%%%%%%%%%%
  %
%%%%%                     G  G  G  G  G  G  G  G  G
 %%%
  %

  %%%%%%%%%%%%%%%%%%%%%%%%%%%%%%%%%%%%%%%%%%%%%%%%%%%%%%%%%%%%%%%%%%%%%%%%%%%%%
  %
%%%%%                     H  H  H  H  H  H  H  H  H
 %%%
  %

  %%%%%%%%%%%%%%%%%%%%%%%%%%%%%%%%%%%%%%%%%%%%%%%%%%%%%%%%%%%%%%%%%%%%%%%%%%%%%
  %
%%%%%                     I  I  I  I  I  I  I  I  I
 %%%
  %

\nomenclature{ispell}{International Ispell é um programa interactivo para
  verificação ortográfica que suporta várias línguas Europeias.
}

  %%%%%%%%%%%%%%%%%%%%%%%%%%%%%%%%%%%%%%%%%%%%%%%%%%%%%%%%%%%%%%%%%%%%%%%%%%%%%
  %
%%%%%                     J  J  J  J  J  J  J  J  J
 %%%
  %

  %%%%%%%%%%%%%%%%%%%%%%%%%%%%%%%%%%%%%%%%%%%%%%%%%%%%%%%%%%%%%%%%%%%%%%%%%%%%%
  %
%%%%%                     K  K  K  K  K  K  K  K  K
 %%%
  %

\nomenclature{Kerberos}{Protocolo de autenticação em
rede. Está desenhado para
providenciar autenticação forte entre aplicações cliente/servidor através de
criptografia de chave secreta.}

  %%%%%%%%%%%%%%%%%%%%%%%%%%%%%%%%%%%%%%%%%%%%%%%%%%%%%%%%%%%%%%%%%%%%%%%%%%%%%
  %
%%%%%                     L  L  L  L  L  L  L  L  L
 %%%
  %

\nomenclature{LDAP}{Lightweight Directory Access
Protocol é um conjunto de protocolos para acesso a
informação organizada em directórios. O \LDAP{} baseia-se na norma
X.500, sendo, no entanto, mais simples e
interoperável com protocolos Internet.}

  %%%%%%%%%%%%%%%%%%%%%%%%%%%%%%%%%%%%%%%%%%%%%%%%%%%%%%%%%%%%%%%%%%%%%%%%%%%%%
  %
%%%%%                     M  M  M  M  M  M  M  M  M
 %%%
  %

\nomenclature{m4}{Processador de macros. \m4{} possui funções internas para inclusão de ficheiros,
  execução de comandos, aritmética, etc.}

  %%%%%%%%%%%%%%%%%%%%%%%%%%%%%%%%%%%%%%%%%%%%%%%%%%%%%%%%%%%%%%%%%%%%%%%%%%%%%
  %
%%%%%                     N  N  N  N  N  N  N  N  N
 %%%
  %

\nomenclature{noweb}{Ferramenta de programação
  literária independente da linguagem de pro\-gra\-ma\-ção.}

  %%%%%%%%%%%%%%%%%%%%%%%%%%%%%%%%%%%%%%%%%%%%%%%%%%%%%%%%%%%%%%%%%%%%%%%%%%%%%
  %
%%%%%                     O  O  O  O  O  O  O  O  O
 %%%
  %

\nomenclature{OWL}{Web Ontology Language. \OWL{} é uma
linguagem que permite a definição de ontologias baseadas na Web para permitir
a integração de dados e a interoperabilidade entre comunidades. \OWL{} parte
de \RDF{} e \RDFS{} e adiciona vocabulário para a descrição de propriedades e
classes: relações entre classes, cardinalidade, igualdade, entre
outros. \OWL{} permite a
definição de ontologias compatíveis com a arquitectura da Web, em geral, e com
a Semantic Web, em particular.}

  %%%%%%%%%%%%%%%%%%%%%%%%%%%%%%%%%%%%%%%%%%%%%%%%%%%%%%%%%%%%%%%%%%%%%%%%%%%%%
  %
%%%%%                     P  P  P  P  P  P  P  P  P
 %%%
  %

\nomenclature{Planeador de frases}{Designação alternativa para o
micro\pdash{}planeador (projecto \RAGS).}

\nomenclature{Planeador de texto}{Designação alternativa para o micro\pdash{}planeador.}

  %%%%%%%%%%%%%%%%%%%%%%%%%%%%%%%%%%%%%%%%%%%%%%%%%%%%%%%%%%%%%%%%%%%%%%%%%%%%%
  %
%%%%%                     R  R  R  R  R  R  R  R  R
 %%%
  %

\nomenclature{RDF}{Resource Definition Framework. \RDF{} é
parte da W3C Metadata Activity (\url{http://www.w3.org/Metadata/}). O
objectivo desta actividade, e do \RDF{} em particular, é a produção de uma
linguagem para o intercâmbio de descrições dos recursos da Web. As descrições
destinam-se a usos
automáticos.}

\nomenclature{RDFS}{\RDF{} Schema. \RDFS{} é uma extensão
semântica do \RDF{}, providenciando mecanismos que permitem a descrição de
grupos de recursos relacionados, bem como as relações entre esses recursos. As
descrições são escritas de acordo com \RDF{}. Os recursos são utilizadas para
determinar as características de outros recursos, tais como domínios e gamas
de propriedades.}

  %%%%%%%%%%%%%%%%%%%%%%%%%%%%%%%%%%%%%%%%%%%%%%%%%%%%%%%%%%%%%%%%%%%%%%%%%%%%%
  %
%%%%%                     S  S  S  S  S  S  S  S  S
 %%%
  %

\nomenclature{StoryBook}{\StoryBook{} é uma implementação da arquitecura de
  geração de prosa narrativa AUTHOR. O sistema
  executa as funções de planeamento da narrativa, assim como as funções de
  geração de língua natural. O texto final é construído utilizando o
  realizador de superfície \FUF/\SURGE. As histórias geradas
  situam-se no domínio do Capuchinho Vermelho~\index{Capuchinho
    Vermelho}\index{Little Red Riding Hood|see{Capuchinho Vermelho}}.}

\nomenclature{SURGE}{Realizador de superfície para Inglês (Systemic
  Unification Realization Grammar of English). Uma apresentação do sistema é
  feita em.}

  %%%%%%%%%%%%%%%%%%%%%%%%%%%%%%%%%%%%%%%%%%%%%%%%%%%%%%%%%%%%%%%%%%%%%%%%%%%%%
  %
%%%%%                     T  T  T  T  T  T  T  T  T
 %%%
  %

  %%%%%%%%%%%%%%%%%%%%%%%%%%%%%%%%%%%%%%%%%%%%%%%%%%%%%%%%%%%%%%%%%%%%%%%%%%%%%
  %
%%%%%                     U  U  U  U  U  U  U  U  U
 %%%
  %

  %%%%%%%%%%%%%%%%%%%%%%%%%%%%%%%%%%%%%%%%%%%%%%%%%%%%%%%%%%%%%%%%%%%%%%%%%%%%%
  %
%%%%%                     U  U  U  U  U  U  U  U  U
 %%%
  %

\nomenclature{vnACCMS}{Sistema que realiza tarefas de segmentação
  de palavras e etiquetação morfológica. O sistema utiliza o formato de
  representação para recursos linguísticos tal como definido no âmbito do
  trabalho da equipa \ISO{} \TCxxxviiSCiv{}. Ver
  \url{http://www.loria.fr/equipes/led/outils.php}.}

  %%%%%%%%%%%%%%%%%%%%%%%%%%%%%%%%%%%%%%%%%%%%%%%%%%%%%%%%%%%%%%%%%%%%%%%%%%%%%
  %
%%%%%                     W  W  W  W  W  W  W  W  W
 %%%
  %

\nomenclature{WordNet}{Léxico semântico para Inglês.}

  %%%%%%%%%%%%%%%%%%%%%%%%%%%%%%%%%%%%%%%%%%%%%%%%%%%%%%%%%%%%%%%%%%%%%%%%%%%%%
  %
%%%%%                     X  X  X  X  X  X  X  X  X
 %%%
  %

\nomenclature{XMI}{\XML{} Metadata Interchange. \XMI{} é um
enquadramento para a definição, intercâmbio, manipulação e integração de
objectos \XML. As normas baseadas em \XMI{} permitem a integração de
ferramentas e repositórios.}

\nomenclature{XML}{Extensible Markup Language é um formato de texto, simples e
flexível, derivado de \SGML{} (\ISO{}~8879).}

\nomenclature{XSD}{\XML{} Schema Definition. Os esquemas \XML{}
expressam vocabulários partilhados e providenciam formas de definir a
estrutura, conteúdo e semântica de documentos \XML. Ver
\url{www.oasis-open.org/cover/schemas.html}.}

\nomenclature{XSLT}{\XSL{} Transformations é uma linguagem para
  transformar documentos \XML. A transformação \XSLT{} descreve as regras para
  transformar uma árvore de entrada numa árvore de saída independente da
  árvore original. A linguagem permite filtrar a árvore original assim como a
  adição de estruturas arbitrárias.}

  %
 %%%
%%%%%                                F   I   M
  %
  %%%%%%%%%%%%%%%%%%%%%%%%%%%%%%%%%%%%%%%%%%%%%%%%%%%%%%%%%%%%%%%%%%%%%%%%%%%%%
